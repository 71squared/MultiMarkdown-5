\input{mmd-article-header}
\def\mytitle{Sample MultiMarkdown Document}
\def\latexmode{memoir}
\def\keywords{MultiMarkdown, Markdown, XML, XHTML, XSLT, PDF}
\def\mycopyright{2011 Fletcher T. Penney.  \\
This work is licensed under a Creative Commons License.  \\
http:/\slash creativecommons.org\slash licenses\slash by-nc-sa\slash 3.0\slash }
\input{mmd-natbib-plain}
\input{mmd-article-begin-doc}

\chapter{Introduction}
\label{introduction}

As I add increasing numbers of features to MultiMarkdown, I decided it was time to create a sample document to show them off. Many of the features are demonstrated in the \href{http://fletcherpenney.net/mmd/users_guide/}{MultiMarkdown User's Guide}\footnote{\href{http://fletcherpenney.net/mmd/users\_guide/}{http:/\slash fletcherpenney.net\slash mmd\slash users\_guide\slash }}, but some are not. 

Additionally, it's easy for those features to get lost within all of the technical documentation. This document is designed to \emph{demonstrate}, not describe, most of the features of MultiMarkdown. 

\input{mmd-memoir-footer}

\end{document}
